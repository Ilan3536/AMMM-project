\documentclass{article}
\usepackage{amsmath}
\usepackage{amsfonts}
\usepackage{amssymb}
\usepackage{geometry}
\usepackage{graphicx}
\usepackage{placeins}
\usepackage{flafter}
\usepackage{xcolor}
\usepackage{listings}
\usepackage{algorithm}
\usepackage{algpseudocode}
\usepackage{hyperref}
\usepackage{enumitem}
\usepackage[affil-it]{authblk}
\geometry{a4paper, margin=1in}

\title{Algorithmic Methods for Mathematical Models\\
 Course Project}
\author{Marco Bartoli, Ilan Vezmarovic \\
    UPC Universitat Politècnica de Catalunya}

\begin{document}

\maketitle

\section{Formal Problem Definition}

\subsection{Inputs}

\begin{itemize}
    \item $n$: number of products
    \item $x$: height of the suitcase in millimeters
    \item $y$: width of the suitcase in millimeters
    \item $c$: limit to the total weight of the suitcase in grams
    \item $p_i$: price of the $i$-th product in euros
    \item $w_i$: weight of the $i$-th product in grams
    \item $s_i$: side length of the $i$-th product's (square) box in millimeters
\end{itemize}

\subsection{Outputs}

\begin{itemize}
    \item Indices of the products chosen to maximize the accumulated price
    \item Arrangement of the products in the suitcase
\end{itemize}

\section{Mathematical Formulation}

\subsection{Decision Variables}

\begin{itemize}
    \item $\text{Chosen}_i$: binary variable that is 1 if object $i$ is chosen, and 0 otherwise.
    \item $\text{PointsX}_i$: the x-coordinate of the bottom-left corner of object $i$.
    \item $\text{PointsY}_i$: the y-coordinate of the bottom-left corner of object $i$.
    \item $\text{Overlap}_{i,j,d}$: binary variable indicating if objects $i$ and $j$ do not overlap in direction $d$, where $d \in \{1, 2, 3, 4\}$.
\end{itemize}

\subsection{Objective Function}

Maximize the total price of the chosen objects:

\[
\text{maximize} \sum_{i=1}^n p_i \cdot \text{Chosen}_i
\]

\subsection{Constraints}

\subsubsection{Max Weight Constraint}

Ensure the total weight of the chosen objects does not exceed the suitcase's capacity:

\[
\sum_{i=1}^n w_i \cdot \text{Chosen}_i \leq c
\]

\subsubsection{Coordinate Bounds Constraints}

Ensure each object lies entirely within the suitcase's boundaries:

\[
\forall i \in \{1, \ldots, n\}, \quad \text{PointsX}_i \geq 1
\]

\[
\forall i \in \{1, \ldots, n\}, \quad \text{PointsY}_i \geq 1
\]

\[
\forall i \in \{1, \ldots, n\}, \quad \text{PointsX}_i + s_i - 1 \leq x
\]

\[
\forall i \in \{1, \ldots, n\}, \quad \text{PointsY}_i + s_i - 1 \leq y
\]

\subsubsection{Non-Overlapping Constraints}

Ensure no two chosen objects overlap within the suitcase using the big-M method:

1. Horizontal non-overlapping to the right:

\[
\forall i, j \in \{1, \ldots, n\}, i \neq j, \quad \text{PointsX}_i - \text{PointsX}_j + s_i \leq -M \cdot (\text{Chosen}_i + \text{Chosen}_j + \text{Overlap}_{i,j,1} - 3)
\]

2. Horizontal non-overlapping to the left:

\[
\forall i, j \in \{1, \ldots, n\}, i \neq j, \quad \text{PointsX}_j - \text{PointsX}_i + s_j \leq -M \cdot (\text{Chosen}_i + \text{Chosen}_j + \text{Overlap}_{i,j,2} - 3)
\]

3. Vertical non-overlapping upwards:

\[
\forall i, j \in \{1, \ldots, n\}, i \neq j, \quad \text{PointsY}_i - \text{PointsY}_j + s_i \leq -M \cdot (\text{Chosen}_i + \text{Chosen}_j + \text{Overlap}_{i,j,3} - 3)
\]

4. Vertical non-overlapping downwards:

\[
\forall i, j \in \{1, \ldots, n\}, i \neq j, \quad \text{PointsY}_j - \text{PointsY}_i + s_j \leq -M \cdot (\text{Chosen}_i + \text{Chosen}_j + \text{Overlap}_{i,j,4} - 3)
\]

\subsubsection{At Least One Not Overlapping Constraint}

Ensure that for any two objects, at least one of the non-overlapping conditions is satisfied:

\[
\forall i, j \in \{1, \ldots, n\}, i \neq j, \quad \sum_{d=1}^4 \text{Overlap}_{i,j,d} \geq 1
\]

\newpage

\section{Heuristic}

\subsection{Greedy Algorithm}
A greedy algorithm makes the optimal choice at each step by selecting products based on $Qvalue = \frac{price}{side \times weight}$.

\subsubsection{Pseudocode}
\begin{algorithm}
\caption{Greedy Algorithm}
\begin{algorithmic}
    \State \textbf{Input:} A problem instance with $n$ products, suitcase dimensions $(x, y)$, and weight capacity $c$.
    \State \textbf{Output:} A feasible solution with selected products that maximize the total price.    

    \State Initialize solution as empty
    \State Sort products by $(price / side / weight)$ in descending order

    \For{each product in sorted products}
        \If{product can fit in the suitcase and does not exceed weight limit}
            \State Add product to solution
            \State Update suitcase dimensions and weight capacity
        \EndIf
    \EndFor

    \State \Return solution
\end{algorithmic}
\end{algorithm}

\section{Meta-heuristics}

\subsection{Local Search Algorithm}
After constructing an initial solution using the greedy algorithm, a local search is applied to improve it by exploring neighboring solutions.

\subsubsection{Pseudocode}
\begin{algorithm}
\caption{Local Search Algorithm}
\begin{algorithmic}
    \State \textbf{Input:} An initial solution, mode (FIRST\_IMPROVEMENT or BEST\_IMPROVEMENT), strategy (EXCHANGE or REASSIGNMENT)
    \State \textbf{Output:} An improved solution

    \State bestSolution = initialSolution
    \State improved = true
        
    \While{improved}
        \State improved = false

        \For{each product in bestSolution}
            \If{strategy == EXCHANGE}
                \State newSolution = exchangeProduct(bestSolution, product)
            \ElsIf{strategy == REASSIGNMENT}
                \State newSolution = reassignProduct(bestSolution, product)
            \EndIf

            \If{newSolution is better than bestSolution}
                \State bestSolution = newSolution
                \State improved = true
                
                \If{mode == FIRST\_IMPROVEMENT}
                    \State \Return bestSolution
                \EndIf
            \EndIf
        \EndFor
    \EndWhile

    \State \Return bestSolution
\end{algorithmic}
\end{algorithm}

\begin{algorithm}
\caption{Exchange Product}
\begin{algorithmic}
    \State \textbf{Input:} A solution, a product to be exchanged
    \State \textbf{Output:} A new solution with the product exchanged

    \State bestSolution = initialSolution
    \State newProductList = copy of the selected product list from the solution
    \State remove the selected product from newProductList

    \For{each unselected product}
        \State add unselected product to newProductList
        \State newSolution = findSolutionForProductList(newProductList)
        
        \If{newSolution is better than bestSolution}
            \State bestSolution = newSolution

            \If{mode == \texttt{FIRST\_IMPROVEMENT}}
                \State \Return bestSolution
            \EndIf
        \EndIf
        \State remove unselected product from newProductList        
    \EndFor

    \State \Return bestSolution
\end{algorithmic}
\end{algorithm}

\begin{algorithm}
\caption{Reassign Product}
\begin{algorithmic}
    \State \textbf{Input:} A solution, a product to be excluded for re-evaluation
    \State \textbf{Output:} A new solution with products reassigned

    \State bestSolution = initialSolution
    
    \For{each selected product}
        \State newProductList = copy of the all product list
        \State remove the selected product from newProductList
        
        \State newSolution = findSolutionForProductList(newProductList)    

        \If{newSolution is better than current solution}
            \State bestSolution = newSolution
    
            \If{mode == FIRST\_IMPROVEMENT}
                \State \Return bestSolution
            \EndIf
        \EndIf                
    \EndFor
    
    \State \Return bestSolution
\end{algorithmic}
\end{algorithm}


\subsection{GRASP (Greedy Randomized Adaptive Search Procedure)}
GRASP consists of a construction phase where a feasible solution is generated using a randomized greedy algorithm, followed by a local search to iteratively improve the solution. By default, GRASP uses \texttt{FIRST\_IMPROVEMENT} and \texttt{EXCHANGE} for local search.

\subsubsection{Pseudocode}
\begin{algorithm}
\caption{GRASP Algorithm}
\begin{algorithmic}
    \State \textbf{Input:} A problem instance, maxIterations, alpha (for RCL threshold)
    \State \textbf{Output:} The best solution found

    \State bestSolution = null

    \For{iteration = 1 to maxIterations}
        \State greedySolution = constructGreedyRandomizedSolution(problem, alpha)
        \State localOptimalSolution = LOCAL\_SEARCH(greedySolution, FIRST\_IMPROVEMENT)

        \If{bestSolution is null or localOptimalSolution is better than bestSolution}
            \State bestSolution = localOptimalSolution
        \EndIf
    \EndFor

    \State \Return bestSolution
\end{algorithmic}
\end{algorithm}

\begin{algorithm}
\caption{Construct Greedy Randomized Solution}
\begin{algorithmic}
    \State \textbf{Input:} A problem instance, alpha (for RCL threshold)
    \State \textbf{Output:} A feasible solution

    \State Initialize solution as empty
    \State Sort products by $(price / side / weight)$ in descending order

    \While{there are remaining products}
        \State $qMax$ = maximum $Q$ value in remaining products
        \State $qMin$ = minimum $Q$ value in remaining products
        \State threshold = $qMax - alpha \times (qMax - qMin)$

        \State RCL = \{products with $Q$ value $>=$ threshold\}
        \State selectedProduct = randomly select a product from RCL

        \If{selectedProduct fits in the suitcase and does not exceed weight limit}
            \State Add selectedProduct to solution
            \State Update suitcase dimensions and weight capacity
        \EndIf
    \EndWhile

    \State \Return solution
\end{algorithmic}
\end{algorithm}


\newpage

\section{Tuning Local Search}

\begin{figure}[!h]
    \centering
    \includegraphics[width=0.9\textwidth]{./documentation/assets/localSearchParams.timeChart.pdf}
    \caption{Time Parameters Local Search}
    \label{fig:local_time}
\end{figure}\FloatBarrier

\begin{figure}
    \centering
    \includegraphics[width=1\textwidth]{./documentation/assets/localSearchParams.objectiveChart.pdf}
    \caption{Objective Parameters Local Search}
    \label{fig:local_objective}
\end{figure}\FloatBarrier

\newpage

\section{Tuning GRASP}

\begin{figure}[!h]
    \centering
    \includegraphics[width=1\textwidth]{./documentation/assets/GRASPParams.timeChart.pdf}
    \caption{Time Parameters GRASP}
    \label{fig:grasp_time}
\end{figure}\FloatBarrier

\begin{figure}
    \centering
    \includegraphics[width=1\textwidth]{./documentation/assets/GRASPParams.objectiveChart.pdf}
    \caption{Objective Parameters GRASP}
    \label{fig:grasp_objective}
\end{figure}\FloatBarrier

\newpage

\section{Performance overall}

\begin{figure}[!h]
    \centering
    \includegraphics[width=1\textwidth]{./documentation/assets/all.timeChart.pdf}
    \caption{Time}
    \label{fig:all_time}
\end{figure}\FloatBarrier

\begin{figure}
    \centering
    \includegraphics[width=1\textwidth]{./documentation/assets/all.objectiveChart.pdf}
    \caption{Objective}
    \label{fig:all_objective}
\end{figure}\FloatBarrier

\section{How to run}

\begin{itemize}
  \item Install IBM CPLEX
  \item Install Java 22
  \item Install Maven
\end{itemize}

Those are the commands to run it from command line without using an IDE

This will run both OPL and Heuristic and output benchmark values in CSV.

\begin{lstlisting}[language=bash]
cd heuristic
mvn compile exec:java -Dexec.mainClass="edu.upc.fib.ammm.Main" -Dexec.args="../opl"
\end{lstlisting}

Make plots of a run

\begin{lstlisting}[language=bash]
mvn compile exec:java -Dexec.mainClass="edu.upc.fib.ammm.Plot" -Dexec.args="output.csv"
\end{lstlisting}

\end{document}